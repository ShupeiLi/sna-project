\documentclass[aspectratio=169]{beamer}

\graphicspath{{graphics/}}

\usetheme[style=fwn]{leidenuniv}
\useinnertheme{circles}
\useoutertheme[subsection=false]{miniframes}
\beamertemplatenavigationsymbolsempty

% uncomment next line to let framesubtitle have palette primary color
%\setbeamercolor{framesubtitle}{use={palette primary},fg=palette primary.bg}

% uncomment next line to remove navigation symbols from the pdf
%\setbeamertemplate{navigation symbols}{}

\title{node2vec: Scalable Feature Learning for Networks}
\subtitle{Authors: Aditya Grover and Jure Leskovec}
\author{Chenyu Shi and Shupei Li}
\institute[LIACS]{Leiden Institute of Advanced Computer Science}
\date{November 18, 2022}


\begin{document}

\begin{frame}[plain]
	\titlepage
\end{frame}

\begin{frame}
	\tableofcontents
\end{frame}

\section{Introduction}
\begin{frame}
    \frametitle{Introduction to Graph Embeddings}
\end{frame}

\section{Related Work}
\begin{frame}
    \frametitle{Related Work}
\end{frame}

\section{Methodology}
\begin{frame}
    
\end{frame}

\section{Our work}
\begin{frame}
    
\end{frame}

\section{Future work}
\begin{frame}
    
\end{frame}

\begin{frame}
	\frametitle{There Is No Largest Prime Number}
	\framesubtitle{The proof uses \textit{reductio ad absurdum}.}
	\begin{theorem}
		There is no largest prime number.
	\end{theorem}
	\begin{proof}
		\begin{enumerate}
			\item<1-| alert@1> Suppose $p$ were the largest prime number.
			\item<2-> Let $q$ be the product of the first $p$ numbers.
			\item<3-> Then $q+1$ is not divisible by any of them.
			\item<1-> But $q + 1$ is greater than $1$, thus divisible by some prime
				number not in the first $p$ numbers.\qedhere
		\end{enumerate}
	\end{proof}
\end{frame}

\begin{frame}
	\frametitle{Block colors}
	\begin{block}{A block}
		With text
	\end{block}
	\begin{alertblock}{An alert block}
		With text
	\end{alertblock}
	\begin{exampleblock}{An example block}
		\begin{itemize}
			\item An item
			\item And another one
		\end{itemize}
	\end{exampleblock}
\end{frame}

\end{document}
