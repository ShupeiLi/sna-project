\documentclass[sigconf]{acmart}
\AtBeginDocument{%
  \providecommand\BibTeX{{%
    \normalfont B\kern-0.5em{\scshape i\kern-0.25em b}\kern-0.8em\TeX}}}
\setcopyright{rightsretained}
\acmConference[SNACS '22]{Social Network Analysis for Computer Scientists Course 2022}{Master CS, Fall 2022}{Leiden, the Netherlands}
\copyrightyear{2022}
\acmYear{2022}
\acmISBN{}
\acmDOI{}
%%%% Do not modify lines 1-11

\begin{document}

\title{Your Original and Relevant Course Project Title}
\subtitle{ Social Network Analysis for Computer Scientists --- Course paper} % do not modify thi

\author{Student name 1}
%\orcid{0000-0001-5468-1030}
\email{student2@umail.leidenuniv.nl}
\affiliation{
  \institution{LIACS, Leiden University}
  \city{Leiden}
  \country{Netherlands}}

\author{Student name 2}
%\orcid{0000-0001-5468-1030}
\email{student2@umail.leidenuniv.nl}
\affiliation{
  \institution{LIACS, Leiden University}
  \city{Leiden}  
  \country{Netherlands}}

\renewcommand{\shortauthors}{Lastname1 and Lastname2}

\keywords{keyword1, keyword2, keyword3, social network analysis, network science}

\begin{abstract}

% the abstract summarizes the entire paper: context, problem, solution, approach, data, experimental results, conclusion and real-world implications. but, shortly, so in half a column or so.

\end{abstract}

\maketitle

\section{Introduction}

% textual description of the context, the problem considered, why it is important, how it is addressed in other works, and what real-world applications are. end with a paragraph on what the contributions of the paper are (so, which problems you solve or which research questions you address), and finally a paragraph on how the remainder of the paper is organized.

\section{Related work}

% briefly discuss other papers related to this work, or previous work describing other approaches for the same problem. end with a statement on how your paper contributes to these works.

\section{Preliminaries}

% necessary notation, formal definitions, and a problem statement.

\section{Approach}

% factually present your solution to the problem studied in the paper. this may include a repetition in your own words of the original paper that you studied. strive to have at least one explanatory picture that explains the approach.

\section{Data}

% what datasets did you use, what types of networks do they represent, where did you obtain the data? did you do any processing? give a table describing data characteristics, such as number of nodes, edges, average degree, etc.

\section{Experiments}

% subsections on for example the experimental setup (which software, hardware and parameters did you choose), as well as the results of applying your approach to the data you described in preceding sections, leading to results that answer your research questions. you likely present some tables and figures

\section{Conclusion}

% summarize in at most one column the main results of the paper, stating how you addressed the problem statement and how the experiments help understand whether  the approach works (or not). end with one or two short suggestions for future work. 

\begin{acks}
% optional: acknowledgments, so people or organizations you wish to thank 
\end{acks}


\bibliographystyle{ACM-Reference-Format}
\bibliography{snacspaper} % put your references in bibtex format in snacspaper.bib

%\appendix
%\section{Robustness checks}
% Appendixes are optional for the course project. they can contain proofs, figures or tables that do not fit in the main body of the text, but are handy as background information

\end{document}
\endinput
